\documentclass[garamond]{goose-article}

\hypersetup{pdfauthor={T.W.J. de Geus}}

%\title{}

%\author[1]{}

%\affil[1]{} % use '\nl' for line-breaks

%\contact{}

%\header{}
%\headerEven{}
%\headerOdd{}

\newcommand\leftstar[1]{\hspace*{-.3em}~^\star\!#1}
\newcommand\ST[1]{\hspace*{-.3em}~^\star\!#1}

\newcommand\T[1]{\bm{{#1}}}

\newcommand\TT[1]{\mathbb{{#1}}}

\begin{document}

% \maketitle

%\begin{abstract}
%\end{abstract}

%\keywords{}

\paragraph{Preface}

% This model is described in \citet{Geers2004}

\paragraph{Tensor definition}

\emph{Finger tensor} (also known the name \emph{left Cauchy-Green deformation})
\begin{equation}
  \T{b} \equiv \T{F} \cdot \T{F}^T
\end{equation}
\emph{Left stretch tensor}:
\begin{equation}
  \T{v} \equiv \sqrt{\T{B}}
\end{equation}
\emph{Hencky’s logarithmic strain tensor}:
\begin{equation}
  \T{\varepsilon} \equiv \ln \T{v} = \tfrac{1}{2} \ln \T{b}
\end{equation}


\paragraph{Model}

This model essentially relies on splitting the deformation gradient $\T{F}$ in an elastic part, $\T{F}_\mathrm{e}$, and a plastic part, $\T{F}_\mathrm{p}$, as follows:
\begin{equation}
  \T{F} \equiv \T{F}_\mathrm{e} \cdot \T{F}_\mathrm{p}
\end{equation}
The model is fully defined in the deformed configuration. To that end it is defined in terms of the logarithmic strain. The Kirchhoff stress $\T{\tau}$ is thereby related to the logarithmic elastic strain $\T{\varepsilon}_\mathrm{e} = \tfrac{1}{2} \ln \T{b}_\mathrm{e}$ in the usual way:
\begin{equation}
  \T{\tau} \equiv K \mathrm{tr} \left( \T{\varepsilon}_\mathrm{e} \right) \T{I} + 2 G \T{\varepsilon}_\mathrm{e}^\mathrm{d}
\end{equation}
where $K$ is the bulk modulus and $G$ is the shear modulus, $\T{I}$ is the second order unit tensor, and $\T{\varepsilon}_\mathrm{e}^\mathrm{d}$ is the deviator of the elastic strain tensor. Equivalently this can be expressed in terms of a fourth order material stiffness
\begin{equation}
  \T{\tau} \equiv \TT{C} : \T{\varepsilon}_\mathrm{e},
  \quad
  \TT{C} = K \T{I} \otimes \T{I} + 2 G \TT{I}^\mathrm{d}
\end{equation}
with $\TT{I}^\mathrm{d} = \TT{I} - \tfrac{1}{3} \T{I} \otimes \T{I}$ the deviator projection. The elastic domain is bounded by a yield function, that, following standard $J_2$ plasticity, reads
\begin{equation}
  \Phi(\T{\tau}, \varepsilon_\mathrm{p}) \equiv \tau_\mathrm{eq} - \tau_\mathrm{y}(\varepsilon_\mathrm{p}) \leq 0
\end{equation}
where
\begin{equation}
  \tau_\mathrm{eq} \equiv \sqrt{ \T{\tau}^\mathrm{d} : \T{\tau}^\mathrm{d} }
\end{equation}
is the equivalent stress; $\tau_\mathrm{y}$ is the equivalent yield stress that may, in the case of hardening, be a function of the plastic strain; $\varepsilon_\mathrm{p}$ is the effective plastic strain which depends on the entire strain history:
\begin{equation}
  \varepsilon_\mathrm{p} = \int\limits_0^t \dot{\gamma} \;\mathrm{d}t^\prime
  \label{eq:history}
\end{equation}
whereby the plastic strain rate, $\dot{\gamma}$, is by construction non-negative. Also note that it is defined in terms of some pseudo-time, as there is no rate-sensitivity. In line with $J_2$-plasticity, normality is used to determined the direction of plastic flow. This corresponds to the following associative flow rule:
\begin{equation}
  \dot{\bm{\varepsilon}}_\mathrm{p} = \dot{\gamma} \bm{N}
  \label{eq:flow-rule}
\end{equation}
where $\bm{N}$ is the Prandtl--Reuss flow vector, which is defined through normality:
\begin{equation}
  \bm{N}
  = \frac{\partial \Phi}{\partial \T{\tau}}
  = \frac{3}{2} \frac{\T{\tau}^\mathrm{d}}{\tau_\mathrm{eq}}
\end{equation}

\paragraph{Trial state}

The path-dependent model is solved by discretising in pseudo-time. Accordingly, the deformation is applied in small steps, transforming Eq.~\eqref{eq:history} in
\begin{equation}
  \varepsilon_\mathrm{p} = \sum \Delta \gamma^t
\end{equation}
Solving proceeds through the formulation of a trial state, whereby the full deformation increment is assumed elastic. From this potentially unphysical state one looks for the increment in plastic strain that gives rise to a physically admissible state.

In particular, the incremental deformation tensor
\begin{equation}
  \T{f} = \T{F}^{t + \Delta t} \cdot \left[ \T{F}^t \right]^{-1}
\end{equation}
is used to define the trial state
\begin{equation}
  \begin{cases}
    \ST{\T{b}}_\mathrm{e} = \T{f} \cdot \T{b}_\mathrm{e}^t \cdot \T{f}^\mathsf{T}
    \\
    \ST{\varepsilon_\mathrm{p}} = \varepsilon_\mathrm{p}^t
  \end{cases}
\end{equation}
which gives rise to some $\ST{\T{\tau}}$ and corresponding $\ST{\tau}_\mathrm{eq}$ and $\ST{\T{N}}$. The trial value for the yield function follows as:
\begin{equation}
  \ST{\Phi} = \ST{\tau}_\mathrm{eq} - \tau_\mathrm{y}(\ST{\varepsilon}_\mathrm{p})
\end{equation}
For any $\ST{\Phi} \leq 0$ the trial state is simply the physical admissible state. Otherwise one has to look for a $\Delta \gamma$ such that $\Phi(\T{\tau},\varepsilon_\mathrm{p}) = 0$.

\paragraph{Return mapping}

Given the plastic strain update $\Delta \gamma$ (whose value will be specified below), following Eq.~\eqref{eq:flow-rule} the actual elastic strain
\begin{equation}
  \T{\varepsilon}_\mathrm{e} = \ST{\T{\varepsilon}}_\mathrm{e} - \Delta \gamma \,\ST{\T{N}}
\end{equation}
(here and below the superscript $t + \Delta t$ have been dropped, which is trusted not to give confusion). Consequently the stress reads
\begin{equation}
  \T{\tau} = \ST{\T{\tau}} - 2 G \Delta \gamma \, \ST{\T{N}}
\end{equation}
which, by construction, only affects the deviatoric part of $\T{\tau}$. The stress deviator can be alternatively expressed as
\begin{equation}
  \T{\tau}^\mathrm{d} = \ST{\T{\tau}}^\mathrm{d} - 3 G \Delta \gamma \frac{\ST{\T{\tau}}^\mathrm{d}}{\ST{\tau}_\mathrm{eq}} = \left( 1 - \frac{3 G \Delta \gamma }{\ST{\tau}_\mathrm{eq}} \right) \ST{\T{\tau}}^\mathrm{d}
\end{equation}
from which the following expression for the equivalent stress can be expressed:
\begin{equation}
  \tau_\mathrm{eq} = \ST{\tau}_\mathrm{eq} - 3 G \Delta \gamma
\end{equation}
It can now be realised that the trial and updated deviatoric stresses are \emph{co-linear}:
\begin{equation}
  \frac{ \T{\tau}_\mathrm{d} }{ \tau_\mathrm{eq} }
  =
  \frac{ \ST{\T{\tau}}_\mathrm{d} }{ \ST{\tau}_\mathrm{eq} }, \quad
  \T{N} = \ST{\T{N}}
\end{equation}

To summarise, the return-map involves:
\begin{equation}
  \begin{cases}
    \T{\tau} = \ST{\T{\tau}} - 2 G \Delta \gamma \T{N}
    \\
    \tau_\mathrm{eq} = \ST{\tau}_\mathrm{eq} - 3 G \Delta \gamma
    \\
    \varepsilon_\mathrm{p} = \ST{\varepsilon}_\mathrm{p} + \Delta \gamma
  \end{cases}
\end{equation}
The plastic strain update $\Delta \gamma$ follows from enforcing the yield function:
\begin{equation}
  \Phi(\T{\tau}, \varepsilon_\mathrm{p}) = \ST{\tau}_\mathrm{eq} - 3 G \Delta \gamma - \tau_\mathrm{y} ( \ST{\varepsilon}_\mathrm{p} + \Delta \gamma ) = 0
  \label{eq:return-map:yield-function}
\end{equation}

\paragraph{Linear hardening}

Linear hardening corresponds to the following evolution of yield stress:
\begin{equation}
  \tau_\mathrm{y} ( \varepsilon_\mathrm{p} ) = \tau_\mathrm{y0} + H \varepsilon_\mathrm{p}
\end{equation}
In this case, enforcing the yield function (as in Eq.~\eqref{eq:return-map:yield-function}) corresponds to:
\begin{equation}
  \ST{\Phi} - 3 G \Delta \gamma + H \Delta \gamma = 0
\end{equation}
Solving for $\Delta \gamma$ results in
\begin{equation}
  \Delta \gamma = \frac{\ST{\Phi}}{H + 3G}
\end{equation}



\end{document}
